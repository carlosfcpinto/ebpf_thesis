\chapter{Introduction}
\label{chap:int}


\section{Introduction}

%%%%% Probably talk here about the necessity of preventing data exfiltration in modern companies,. Why the need for security is massive, and the use of eBPF makes sense here.
Data exfiltration is the theft or unauthorized transfer of data from a device or network. This can occur in several ways, and the target data can vary from user credentials, intellectual property and company secrets.
The most common definition of data exfiltration is the unauthorized removal or movement of any data from a device.

The need for the prevention of data exfiltration is particularly pertinent in corporate settings, as there is the potential for the massive loss of revenue by companies which fall victim to these attacks.

Extended Berkeley Packet Filter (eBPF) presented itself as the right tool to avoid these kind of attacks, since it is a technology that allows for the programming of the Linux kernel for networking, observability, tracing and security. It can effectively monitor the entirety of the system, as it runs inside the kernel, and detect data exfiltration attempts.

In this report the research conducted during the internship at Scalabit will be presented, which aimed to prove that eBPF would be a feasible solution to the problem of data exfiltration. Throughout the document, the experiments conducted will be presented and discussed. The purpose of this internship was that of exploring both eBPF from a monitoring and security perspective.

\section{Motivation}

The main objective of this internship was the study and development of an eBPF application that would prevent data exfiltration across a fleet of machines. This opportunity arose when at Scalabit there was a need to study the feasibility of this solution.
The problem in case is a machine that is provided to customer with root access.
What was proposed in this report was to see if it was possible to use eBPF to avoid root access to certain files and folders.
Sometimes it's not simple to update a machine that is making money at customers and change the software or even add security features that impact their way of working.
Doing this with eBBF would prove incredibly handy, being that eBPF is particularly useful, since the classical approach of static configurations, although robust when configured statically during the provisioning of the machine, become brittle when deployed across a fleet of machines and in the face of changing policies.

This application if feasible, would serve the purpose of restricting services and capabilities inside the kernel, based on a per user or service approach, effectively armoring the system from unwanted accesses, either to services or files, thus preventing data exfiltration from said machine.

It presents itself as quite a pertinent problem, both from and academic and business position, since the current approaches to tackle this problem are quite limited.
%eBPF programs are event-driven and are run when the kerne
%eBPF is the right tool for the job in this case, since currently it presents itself as a great monitoring tool. Being that eBPF runs inside the kernel, it has the ability to monitor all of the system calls made on that machine, and therefore detect, and potentially stop those data exfiltration attempts.
%%% Complete this with more information

\section{Scalabit}
Scalabit's objective is helping software companies to achieve outstanding levels when it comes to the software delivery process. Scalabit focuses on continuous integration/continuous delivery in order to deliver more frequently but also with less impact on the field. The lemma "fail fast recover faster" makes a lot of sense for the company.
Given the current panorama worldwide, the security aspect of software is making a lot of impact.
Multiple systems trust their daily operations in software. You can think on medical devices, nuclear power plants, your smartwatch, etc.
We are surrounded by software.
Security of these systems is a critical point. In fact the EU is already looking at the security aspect of software. That's why, in 15th of September 2022, the EU decided to release the Cyber Resilience Act.
This is an EU regulation proposed for improving cybersecurity and cyberresilience through common cybersecurity standards for products with digital elements in the EU, such as required incident reports and automatic security updates.
The challenge here is huge since there is software in the field older than 10 years.
As you can imagine this software is difficult to update and also difficult to adapt.
That's why this project appeared. When, at Scalabit, we started looking at eBPF we though about the option of using the kernel to restrict certain calls and with this increase the level of security of certain systems.
Knowing that this is not the perfect solution, neither the perfect way to do this, there was also the knowledge that the real world is more complicated than it looks like. To achieve this solution ould prove to be a great asset to the company and its clients.


\section{The Problem}
The data exfiltration definition more relevant to this dissertation is that of data exportation and extrusion, posing serious problems for organizations. Failing to control information security could mean the loss of intellectual property or cause reputational and financial damage to an organization.

The problem being faced was that of preventing data exfiltration inside machines being provisioned by a company to clients, which were then in turn trying to access intelectual property, such as software, that were to be private to the company that made said machines.
The data exfiltration problem has seen various approaches to being solved, such as static configurations, which work best when configured statically during the provisioning of a machine, becoming brittle to operate when deployed across a fleet of machines and in the face of changing policies.

One of the many methods that are studied when preventing data exfiltration is that of data hiding, which is the one concerning this investigation. Data hiding can be used as a method for the prevention of data exfiltration by concealing sensitive information. One example of data hiding is steganography, which involves the concealment of sensitive information within another message or file to avoid detection \cite{steg}.


\section{Methodology \& Proposed Solution}
The proposed solution to study aimed to tackle the challenges mentioned in the previous section by providing a method of easily loading configurations with the intetion of preventing data exfiltration, leakage or theft. To achieve this solution, it was chosen to leverage the ease of use and flexibility of eBPF as the pillar of the application itself. It presented many benefits, including the fact that it would not imply the change and consequent rewrite of application code, but it would work as a simple patch on the kernel.

The main objective of this internship was the study of the feasability of an eBPF application preventing data exfiltration, such that one application could be extended across various machines. As stated above, the classical approach of static configuratins, although effective when configured statically during the provisioning of the machines, become hard to operate when deployed across a fleet of said machines.

The implementation of this application serves the purpose of restricting certain system calls inside the kernel, on a per user approach, armoring certain files from unwanted accesses, thus preventing data exfiltration from said files.

As of the study and implementation of this solution the only tool on the market that was similar to it was Tetragon \cite{tetragon}, thus making this solution both an academic and business opportunity, seeing that in some cases the application developed goes above and beyond Tetragon, but it is not without its drawbacks.


\section{Document Organization}

This document is organized as follows:
\begin{enumerate}
	\item \textbf{Core Concepts and State of the Art} - This chapter aims to provide an overview of the Core Concepts necessary to understand the inner workings of eBPF, as well as the Linux kernel.
	\item \textbf{Case Study} - In this chapter the problem to be solved will be described, as well as the proposed solution to solve it. In it there will also be an elaboration on the preliminary experiments conducted as to ascertain that eBPF was the right path in order to try to solve the problem being faced.
	\item \textbf{Tool Development} - In this chapter the development of the final tool aimed to prevent data exfiltration will be present, in it the application choices will be presented, as well as an overall view of the application itself.
	\item \textbf{Use Case} - This chapter serves the purpose of providing an overall view on the expected use cases of the application, as well as the tests ran on said application, as to ensure that the behaviour would be as expected.
	\item \textbf{Critical Analysis} - In this chapter, a critical analysis is provided, concerning the limitations of the application and the overall work done throughout this internship.
	\item \textbf{Conclusion} - This chapter serves the purpose of providing an overall conclusion to the document and the topics approached during this work.
\end{enumerate}

